\section{Design}
\author{Jesper}
\begin{frame}{Design}
	\begin{itemize}
		\item{Reading/Decoding MP3}
		\item{Playing audio synchronously}
		\item{Network communication}
		\item{Establishing connection}
	\end{itemize}
\end{frame}

\section{Decoding}
\subsection{Assumption}

\begin{frame}{Decoding}
	\framesubtitle{Assumption}
	Fast reading and decoding of MP3 data isn't feasible in pure Java
\end{frame}

\subsection{JNI}

\begin{frame}{Decoding}
	\framesubtitle{JNI to the rescue}
	JNI let's us bind \textbf{native} libraries to the Java executable, and
	invoke procedures in them.
\end{frame}

\begin{frame}{Decoding}
	\framesubtitle{JNI to the rescue}
	\centering
	\begin{tikzpicture}[
			->,>=stealth',
			auto,
			node distance=1cm,
			thick,
			main node/.style={rectangle, draw}
		]

		\node[main node] (1) {Caller};
		\node[main node] (2) [right = of 1] {MP3Decoder};
		\node[main node] (3) [below = of 2] {MP3File};
		\node[main node] (4) [right = of 2] {Native library};

		\path[draw, every node/.style={font=\sffamily\small}]
			(1) edge[bend left] node [] {Open file} (2)
			(2) edge[bend left] node [] {Return MP3File} (1)
			(2) edge[dashed] node [] {Instantiates} (3)
			(2) edge[] node [] {Uses} (4)
			(3) -| node [pos=.25, below] {Uses} (4);
	\end{tikzpicture}
\end{frame}

\begin{frame}{Decoding}
	\framesubtitle{JNI to the rescue}
	\centering
	\begin{tikzpicture}[
			->,>=stealth',
			auto,
			node distance=1cm,
			thick,
			main node/.style={rectangle, draw}
		]

		\node[main node] (1) {Caller};
		\node[main node] (3) [right = of 1] {MP3File};
		\node[main node] (2) [above = of 3] {MP3Decoder};
		\node[main node] (4) [right = of 2] {Native library};

		\path[draw, every node/.style={font=\sffamily\small}]
			(1) edge[bend left] node [] {Decode frame} (3)
			(3) edge[bend left] node [] {Return frame data} (1)
			(2) edge[] node [] {Uses} (4)
			(3) -| node [pos=.25, below] {Uses} (4);
	\end{tikzpicture}
\end{frame}

\begin{frame}{Decoding}
	\framesubtitle{MPG123}
	\begin{itemize}
		\item{Fast decoding}
		\item{Well supported (Last release 2 weeks ago)}
	\end{itemize}
\end{frame}

\subsection{Realization}

\begin{frame}{Decoding}
	\framesubtitle{An inconvenient truth}
	While developing the later parts of the system we happened upon the
	android class called \textbf{MediaCodec}, which would have solved
	the problem.
\end{frame}

\section{Playback}
\subsection{Observation}

\begin{frame}{Playback}
	\framesubtitle{Observation}
	Other apps seem to \textbf{drift}, which would indicate a difference
	of playback rate on different devices. We will use \textbf{active}
	synchronization to stay synchronized.
\end{frame}

\subsection{Solution}
\begin{frame}[fragile]{Playback}
	\framesubtitle{The idea}
	\begin{figure}
		\centering
		\begin{tikzpicture}[%
				node distance=0,
				every node/.style = {%
					inner sep=0,
					outer sep=0,
				},
				timeline/.style = {%
					minimum height = 10mm,
					minimum width = #1,
				},
				occurance/.style = {%
					minimum height = 12mm,
					minimum width = 0.5mm,
					fill=black,
				},
				textlabel/.style = {%
				}
			]
			\node[timeline = 15mm, fill=black!20] (b1) [] {};
			\node[timeline = 15mm, fill=black!10] (b2) [right=of b1] {};
			\node[timeline = 15mm, fill=black!20] (b3) [right=of b2] {};
			\node[timeline = 15mm, fill=black!10] (b4) [right=of b3] {};
			\node[occurance] (t0) [left = of b1] {};
			\node[occurance] (t1) [right = of b1] {};
			\node[occurance] (t2) [right = of b2] {};
			\node[occurance] (t3) [right = of b3] {};
			\node[fit={(b1) (b2) (b3) (b4)}, draw=black] (bc) {};

			\node[textlabel] (t0l) [below = 5mm of t0] {0};
			\node[textlabel] (t1l) [below = 5mm of t1] {144};
			\node[textlabel] (t2l) [below = 5mm of t2] {288};
			\node[textlabel] (t3l) [below = 5mm of t3] {432};

			\draw[-] (t0l) -- (t0);
			\draw[-] (t1l) -- (t1);
			\draw[-] (t2l) -- (t2);
			\draw[-] (t3l) -- (t3);
		\end{tikzpicture}
	\end{figure}
\end{frame}

\begin{frame}[fragile]{Playback}
	\framesubtitle{Reality bites back}
	\begin{figure}
		\centering
		\begin{tikzpicture}[%
				node distance=0,
				every node/.style = {%
					inner sep=0,
					outer sep=0,
				},
				timeline/.style = {%
					minimum height = 10mm,
					minimum width = #1,
				},
				occurance/.style = {%
					minimum height = 12mm,
					minimum width = 0.5mm,
					fill=black,
				},
				textlabel/.style = {%
				}
			]
			\node[timeline = 15mm, fill=black, opacity=.2] (b1) [] {};
			\node[timeline = 15mm, fill=black, opacity=.1] (b2) [right=-5mm of b1] {};
			\node[timeline = 15mm, fill=black, opacity=.2] (b3) [right=2mm of b2] {};
			\node[timeline = 15mm, fill=black, opacity=.1] (b4) [right=3mm of b3] {};
			\node[occurance] (t0) [left = of b1] {};
			\node[occurance] (t1) [left = of b2] {};
			\node[occurance] (t2) [left = of b3] {};
			\node[occurance] (t3) [left = of b4] {};
			\node[fit={(b1) (b2) (b3) (b4)}, draw=black] (bc) {};

			\node[textlabel] (t0l) [below = 5mm of t0] {0};
			\node[textlabel] (t1l) [below = 5mm of t1] {144};
			\node[textlabel] (t2l) [below = 5mm of t2] {288};
			\node[textlabel] (t3l) [below = 5mm of t3] {432};

			\draw[-] (t0l) -- (t0);
			\draw[-] (t1l) -- (t1);
			\draw[-] (t2l) -- (t2);
			\draw[-] (t3l) -- (t3);
		\end{tikzpicture}
	\end{figure}
\end{frame}

\subsection{Approximating}
\begin{frame}{Playback}
	\framesubtitle{Observation}
	The mismatches are often below 1000 bytes ($5.6ms$) \textbf{per
	frame}, and tend average out around 0.
\end{frame}

\begin{frame}[fragile]{Playback}
	\framesubtitle{Fuzz factor}
	\begin{figure}
	\centering
		\begin{tikzpicture}[%
				node distance=0,
				every node/.style = {%
					inner sep=0,
					outer sep=0,
				},
				timeline/.style = {%
					minimum height = 40mm,
					minimum width = #1,
				},
				region/.style = {%
					timeline = #1,
					draw,
					dashed,
					fill = none,
				},
				occurance/.style = {%
					minimum height = 12mm,
					minimum width = 0.5mm,
					fill=black,
				},
				textlabel/.style = {%
				}
			]
			\node[timeline = 25mm, fill=black, opacity=.1] (b1) [] {};
			\node[timeline = 25mm, fill=black, opacity=.1] (b2) [right=5mm of b1] {};
			\node[region = 15mm] (fuzz) at (b1.east) {};
			\node[fit={(b1) (b2)}, draw=black] (bc) {};

			\draw[decoration={brace, mirror, raise = 1pt, amplitude=8pt}, decorate, thick]
				(fuzz.south west) -- node[below=10pt] {fuzz factor} (fuzz.south east);

		\end{tikzpicture}
	\end{figure}
\end{frame}

\subsection{Realization}
\begin{frame}{Playback}
	\framesubtitle{Realization}
	Our discovery of MediaCodec led us to \textbf{MediaSync}, which
	would do a lot of these things for us, including resampling and
	scheduling audio. Unfortunately we discovered it too late.
\end{frame}

\section{Synchronization}
\subsection{Observation}

\begin{frame}{Synchronization}
	\framesubtitle{Observation}
	Content synchronisation is \textbf{largely} a matter of agreeing on
	some \textbf{wall-time}, and using that as the authoritative source.
\end{frame}

\subsection{Algorithm}

\begin{frame}{Synchronization}
	\framesubtitle{Synchronization algorithm}
	\centering
	\begin{tikzpicture}[
			->,>=stealth',
			auto,
			node distance=0.5cm,
			thick,
			main node/.style={rectangle, draw}
		]

		\node[main node] (c1) {$C_1$};
		\node[main node] (c2) [above = of c1] {$C_2$};

		\node[main node] (t0) [label={below:$t_0$}, right = of c1] {};
		\node[main node] (t1) [label={$t_1$}, right = 1.5cm of c2] {};
		\node[main node] (t2) [label={$t_2$}, right = of t1] {};
		\node[main node] (t3) [label={below:$t_3$}, right = 2.5cm of t0] {};

		\node[] (c1e) [right = 5cm of c1] {};
		\node[] (c2e) [right = 5cm of c2] {};

		\path[draw, every node/.style={font=\sffamily\small}]
			(c1) edge[-] node [] {} (t0)
			(t0) edge[-] node [] {} (t3)
			(t3) edge[-] node [] {} (c1e);
		\path[draw, every node/.style={font=\sffamily\small}]
			(c2) edge[-] node [] {} (t1)
			(t1) edge[-] node [] {} (t2)
			(t2) edge[-] node [] {} (c2e);

		\path[draw, every node/.style={font=\sffamily\small}]
			(t0) edge[-] node [] {} (t1)
			(t2) edge[-] node [] {} (t3);
	\end{tikzpicture}
	\vspace{1cm}

	$\theta = \frac{(t_1 - t_0) + (t_2 - t_3)}{2}$
\end{frame}

\begin{frame}{Synchronization}
	\framesubtitle{Going back to the local network}
	NTP provides synchronization provides accurate time synchronizations
	with an \textbf{external} authoritative wall-clock. We just need
	\textbf{local consensus}.
\end{frame}
